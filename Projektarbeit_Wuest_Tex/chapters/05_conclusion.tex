\chapter{Conclusion}

This project set out to answer the research question: 
\begin{center}
    \textit{``How can process mining be utilized to optimize Lead Management and enhance sales performance at Robert Bosch GmbH?''} \\
\end{center}
Based on the conducted analysis of the Bosch Home Comfort Lead Management process, the results of chapters four and five show that Process Mining is a powerful and suitable approach to achieve both objectives.

By combining the two data sources Google Analytics wizard data and the internal Lead Management tool into a unified event log and analyzing it with Celonis, this study makes previously unknown process behavior transparent. The discovery phase reveals that Bosch’s Lead Management process matches the characteristics of a highly unorganized \textbf{spaghetti process}, as described by \citeauthor{AalstSpaghetti} in literature. Traditional reporting tools such as PowerBI are not able to capture this complexity, whereas Process Mining enables an view of actual process executions with building an integrated process model (\ref{fig:wizard_lmt_combined}) that reflects reality.

Regarding \textbf{Lead Management optimization}, Process Mining reveals multiple improvement opportunities within the process. First, the control-flow and variant analyses uncover typical and atypical lead paths, including a dominant ``happy path'' for air conditioning and numerous low-frequency variants that collectively account for more than half of all cases. Second, the throughput-time and drop-off analysis identifies critical points in wizard data, such as questions with unusually high median response times (requested services) and high drop-off rates. These findings lead to instant improvements, for example the removal of a question that causes early termination (question about placement of outdoor unit with $19\%$ drop-off).

Also, long waiting times between lead acceptance and the transition to status “in progress” for heat pump technology become visible. A significant share of leads never reaches the `ìn progress'' status at all.

 In terms of \textbf{enhancing sales performance}, the analysis showed how Process Mining can link process behavior to business outcomes. By combining lead attributes with \ac{kpi}s such as acceptance time and sales rate, the study identified patterns that otherwise would remain hidden. For instance, differences in acceptance time and conversion rates across technologies and installers highlight opportunities for differentiated lead routing and lead prioritization. Furthermore, the use of the Insight Explorer demonstrated how changes, such as making certain wizard fields mandatory or focusing on high-performing building types like ``detached flat'', can  improve sales-related \ac{kpi}s nearly $2\%$ and result in revenue growth (\ref{fig:celonis_insight_explorer}).

As this work focused on a proof of concept for the German market, the findings show high potential for a wider roll-out managed by a Bosch internal IT department (\ac{bdo}) in other countries that participate in Lead Management. 

\textbf{Critical Appraisal}

While this project papers shows the potential of Process Mining inside Lead Management, especially the \ac{llcm} showed limitations while creating the integrated process model. Since the Bosch-internal \ac{lmt} tool is under constant development and multiple parties (up to three different installers and end customers) are involved and can trigger processes at different times, there are more than $2.000$ different variants found by Celonis. Even with combining the $10$ most frequent versions, fitness remains low with $31\%$.

Also, results of this study should be interpreted with caution regarding generalization. The analysis focuses on the German market and a specific implementation of Bosch Home Comfort’s Lead Management process.